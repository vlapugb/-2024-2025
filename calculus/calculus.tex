\documentclass{article}
\usepackage[utf8]{inputenc}
\usepackage[russian]{babel}
\usepackage{epigraph}
\usepackage{amsfonts}
\usepackage{mathtools}
\usepackage{amsmath}
\usepackage{hyperref}

\hypersetup{
	colorlinks=true}

\begin{document}
	\begin{obeylines}
	\begin{titlepage}
		\vspace*{\stretch{1.0}}
		\begin{center}
			\Huge\textbf{Математический анализ}\\
			\bigskip
			\bigskip
			\Large\textbf{Конспект лекций А. С. Роткевича, 2024}
		\end{center}
		\vspace*{\stretch{2.0}}
		\end{titlepage}
		
		\epigraph{--- В библиотеке Калифорнийского университета (UCLA) на одной из полок хранится весьма уникальный том. Вполне вероятно, самиздат, но сплетенный и оформленный очень качественно — он ничем не уступает другим книгам библиотеки в этом отношении. Называется так:
			"Секс, криминал и функциональный анализ. Часть I: Функциональный анализ", J.D. Stein. \\
				  --- Как сказал другой человек: "у этой книги две части, одна написана, другая нет, та, что не написана, важнее"}{\textbf{Из канала \href{https://t.me/mathmemories}{воспоминания математиков}}}
				  
	\section*{\centeringЕще немного про отображения}
	\underline{\textbf{Опр.}} $\Gamma_{f} \subset X \times Y$ - график отображения, удовлетворяющий следующим свойством: \\
	1) $\forall x \in X \ \exists y \in Y: (x, y) \in \Gamma_{f}$ ;
	2) $(x_{1}, y_{1}), (x_{1}, y_{2}) \in \Gamma_{f} \Rightarrow y_{1} = y_{2}$.
	
	\bigskip
	
	\textbf{Упражнения:}
	Доказать, что для $f:X \rightarrow Y$:
	1) $\forall A_{1}, A_{2} \subset X$:
	$f(A_{1} \cap A_{2}) \subset f(A_{1}) \cap f(A_{2})$;
	$f(A_{1} \cup A_{2}) = f(A_{1}) \cup f(A_{2})$.
	2) $\forall B_{1}, B_{2} \subset Y$:
	$f^{-1}(B_{1} \cap B_{2}) = f^{-1}(B_{1}) \cap f^{-1}(B_{2})$;
	$f^{-1}(B_{1} \cup B_{2}) = f^{-1}(B_{1}) \cup f^{-1}(B_{2})$.
	
	\bigskip
	Решения упражнений:
	\bigskip
	\newpage
	
	1)
	$\cap$: пусть $x_{1} \in A_{1}, \ x_{2} \in A{2}, \ x_{1} \neq x_{2},$
	$f(x_{1}) = f_(x_{2}), \ f^{-1}(x_{1}) = \{x_{1}, x_{2}\}$. Тогда очевидно, что \\ $(f(x_{1}) \in f(A_{1}) \cap f(A_{2})) \land \neg (f(A_{1} \cap A_{2}))$;
	$\cup: y \in f(A_{1} \cup A_{2}) \ \Leftrightarrow \ \exists \ x \in A_{1} \lor x \in A_{2}: f(x) = y \ \Leftrightarrow \ y \in f(A_{1}) \lor y \in f(A_{2})$.
	2)
	$\cap: x \in f^{-1}(B_{1}) \cap f^{-1}(B_{2}) \ \Leftrightarrow \ f(x) \in B_{1} \cap B_{2} \ \Leftrightarrow \ x \in f^{-1}(B_{1} \cap B_{2})$;
	$\cup: x \in f^{-1}(B_{1}) \cup f^{-1}(B_{2}) \ \Leftrightarrow \ f(x) \in B_{1} \cup B_{2} \ \Leftrightarrow \ x \in f^{-1}(B_{1} \cup B_{2})$.
	
	\bigskip
	
	\underline{\textbf{Опр.}} Отображение $f: \mathbb{N} \rightarrow X$ называется последовательностью элементов множества X.
	
	\section*{\centeringПоле вещественных чисел}
	\underline{\textbf{Опр.}} Полем $(F, +, *)$ называется множество с двумя заданными на нем бинарными операция со следующими свойствами:
	1) Множество является абелевой группой относительно обеих операций (соблюдаются коммутативность, ассоциативность, существование обратных и нейтральных элементов)
	2) $\forall x, y, z \in F: (x + y)z = xz + yz$.
	
	\textbf{Упражнение:} Доказать, что нейтральный элемент по сложению не равен нейтральному по умножению в нетривиальном поле.
	
	\textbf{Решение:}
	От противного: пусть $a \in F, \ a = a \ \Leftrightarrow \ 0 * a = a \ \Rightarrow \ 0 * a + 0 = a + 0 \Rightarrow \ 0 * (a + 1) = a \ \Rightarrow \ a + 1 = a$ --- противоречие.
	
	\bigskip
	
	\underline{\textbf{Опр.}} S --- множество. Порядок на S --- отношение строгого линейного порядка на S, обозначающееся $<$, со следующими свойствами:
	1) $\forall x, y \in S$ верно только одно: $x < y, x = y, y < x$;
	2) $\forall x, y, z \in S: (x < y \land y < z) \Rightarrow (x < z)$.
	\underline{\textbf{Опр.}} (F, +, *, <) --- упорядоченное поле, если (F, +, *) --- поле и для < верно:
	1) $\forall x, y, z \in F: x < y \Rightarrow x + z < y + z$;
	2) $\forall x, y \in F: x > 0 \land y > 0 \Rightarrow xy > 0$.
	
	Теперь введем определение супремумв и инфимума:
	\underline{\textbf{Опр.}} Пусть $A \subset F, \alpha \in F$ называется супремумом A, если:
	1) $\forall x \in F: x \leq \alpha$;
	2) $\forall \varepsilon > 0 \ \exists x \in A: x + \varepsilon > \alpha$.
	Аналогично определяется инфимум --- как точная нижняя грань множества A.
	
	\underline{\textbf{Опр.}} Упорядоченное поле обладает свойством точной верхней грани, если любое ограниченное сверху множеству имеет супремум.
	
	\smallskip
	
	\underline{\textbf{Теорема}}(без доказательства) Существует единственное упорядоченное поле со свойством точной верхней грани. Оно называется полем вещественных чисел и обозначается как $\mathbb{R}$.
	
	\smallskip
	
	Покажем, что $(\mathbb{Q}, +, *)$ не обладает свойством точной верхней грани.
	\textbf{Д-во:}
	Пусть $A = \{q \in \mathbb{Q} \ | \ q > 0 \land q^{2} < 2\}, \ B = \{q \in \mathbb{Q} \ | \ q > 0 \land q^{2} > 2\}$. Несложно доказать, что $\sqrt{2} \notin \mathbb{Q}$ (от противного) $\Rightarrow \mathbb{Q_{+}}$ разбивается A и B. Пусть $\alpha = sup \ A$. Рассмотрим два случая:
	1) $\alpha \in B:$
	$r = \alpha - \frac{\alpha^{2} - 2}{\alpha + 2} = \frac{2(\alpha - 1)}{\alpha + 2}$;
	$r^{2} - 2 = \frac{2\alpha^{2} - 4}{(\alpha + 2)^{2}} > 0\Rightarrow$ это противоречит определению $\alpha$ как супремума A.
	2) $\alpha \in A:$
	$r = \alpha + \frac{2 - \alpha^{2}}{\alpha + 2} > \alpha$;
	$2 - r^{2} = \frac{4 - 2\alpha^{2}}{(\alpha + 2)^{2}} > 0 \ (\alpha^{2} < 2) \Rightarrow$ вновь противоречие определению $\alpha$ как супремума A.
	\end{obeylines}
\end{document}
			
